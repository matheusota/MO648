
% Default to the notebook output style

    


% Inherit from the specified cell style.




    
\documentclass[11pt]{article}

    
    
    \usepackage[T1]{fontenc}
    % Nicer default font (+ math font) than Computer Modern for most use cases
    \usepackage{mathpazo}

    % Basic figure setup, for now with no caption control since it's done
    % automatically by Pandoc (which extracts ![](path) syntax from Markdown).
    \usepackage{graphicx}
    % We will generate all images so they have a width \maxwidth. This means
    % that they will get their normal width if they fit onto the page, but
    % are scaled down if they would overflow the margins.
    \makeatletter
    \def\maxwidth{\ifdim\Gin@nat@width>\linewidth\linewidth
    \else\Gin@nat@width\fi}
    \makeatother
    \let\Oldincludegraphics\includegraphics
    % Set max figure width to be 80% of text width, for now hardcoded.
    \renewcommand{\includegraphics}[1]{\Oldincludegraphics[width=.8\maxwidth]{#1}}
    % Ensure that by default, figures have no caption (until we provide a
    % proper Figure object with a Caption API and a way to capture that
    % in the conversion process - todo).
    \usepackage{caption}
    \DeclareCaptionLabelFormat{nolabel}{}
    \captionsetup{labelformat=nolabel}

    \usepackage{adjustbox} % Used to constrain images to a maximum size 
    \usepackage{xcolor} % Allow colors to be defined
    \usepackage{enumerate} % Needed for markdown enumerations to work
    \usepackage{geometry} % Used to adjust the document margins
    \usepackage{amsmath} % Equations
    \usepackage{amssymb} % Equations
    \usepackage{textcomp} % defines textquotesingle
    % Hack from http://tex.stackexchange.com/a/47451/13684:
    \AtBeginDocument{%
        \def\PYZsq{\textquotesingle}% Upright quotes in Pygmentized code
    }
    \usepackage{upquote} % Upright quotes for verbatim code
    \usepackage{eurosym} % defines \euro
    \usepackage[mathletters]{ucs} % Extended unicode (utf-8) support
    \usepackage[utf8x]{inputenc} % Allow utf-8 characters in the tex document
    \usepackage{fancyvrb} % verbatim replacement that allows latex
    \usepackage{grffile} % extends the file name processing of package graphics 
                         % to support a larger range 
    % The hyperref package gives us a pdf with properly built
    % internal navigation ('pdf bookmarks' for the table of contents,
    % internal cross-reference links, web links for URLs, etc.)
    \usepackage{hyperref}
    \usepackage{longtable} % longtable support required by pandoc >1.10
    \usepackage{booktabs}  % table support for pandoc > 1.12.2
    \usepackage[inline]{enumitem} % IRkernel/repr support (it uses the enumerate* environment)
    \usepackage[normalem]{ulem} % ulem is needed to support strikethroughs (\sout)
                                % normalem makes italics be italics, not underlines
    

    
    
    % Colors for the hyperref package
    \definecolor{urlcolor}{rgb}{0,.145,.698}
    \definecolor{linkcolor}{rgb}{.71,0.21,0.01}
    \definecolor{citecolor}{rgb}{.12,.54,.11}

    % ANSI colors
    \definecolor{ansi-black}{HTML}{3E424D}
    \definecolor{ansi-black-intense}{HTML}{282C36}
    \definecolor{ansi-red}{HTML}{E75C58}
    \definecolor{ansi-red-intense}{HTML}{B22B31}
    \definecolor{ansi-green}{HTML}{00A250}
    \definecolor{ansi-green-intense}{HTML}{007427}
    \definecolor{ansi-yellow}{HTML}{DDB62B}
    \definecolor{ansi-yellow-intense}{HTML}{B27D12}
    \definecolor{ansi-blue}{HTML}{208FFB}
    \definecolor{ansi-blue-intense}{HTML}{0065CA}
    \definecolor{ansi-magenta}{HTML}{D160C4}
    \definecolor{ansi-magenta-intense}{HTML}{A03196}
    \definecolor{ansi-cyan}{HTML}{60C6C8}
    \definecolor{ansi-cyan-intense}{HTML}{258F8F}
    \definecolor{ansi-white}{HTML}{C5C1B4}
    \definecolor{ansi-white-intense}{HTML}{A1A6B2}

    % commands and environments needed by pandoc snippets
    % extracted from the output of `pandoc -s`
    \providecommand{\tightlist}{%
      \setlength{\itemsep}{0pt}\setlength{\parskip}{0pt}}
    \DefineVerbatimEnvironment{Highlighting}{Verbatim}{commandchars=\\\{\}}
    % Add ',fontsize=\small' for more characters per line
    \newenvironment{Shaded}{}{}
    \newcommand{\KeywordTok}[1]{\textcolor[rgb]{0.00,0.44,0.13}{\textbf{{#1}}}}
    \newcommand{\DataTypeTok}[1]{\textcolor[rgb]{0.56,0.13,0.00}{{#1}}}
    \newcommand{\DecValTok}[1]{\textcolor[rgb]{0.25,0.63,0.44}{{#1}}}
    \newcommand{\BaseNTok}[1]{\textcolor[rgb]{0.25,0.63,0.44}{{#1}}}
    \newcommand{\FloatTok}[1]{\textcolor[rgb]{0.25,0.63,0.44}{{#1}}}
    \newcommand{\CharTok}[1]{\textcolor[rgb]{0.25,0.44,0.63}{{#1}}}
    \newcommand{\StringTok}[1]{\textcolor[rgb]{0.25,0.44,0.63}{{#1}}}
    \newcommand{\CommentTok}[1]{\textcolor[rgb]{0.38,0.63,0.69}{\textit{{#1}}}}
    \newcommand{\OtherTok}[1]{\textcolor[rgb]{0.00,0.44,0.13}{{#1}}}
    \newcommand{\AlertTok}[1]{\textcolor[rgb]{1.00,0.00,0.00}{\textbf{{#1}}}}
    \newcommand{\FunctionTok}[1]{\textcolor[rgb]{0.02,0.16,0.49}{{#1}}}
    \newcommand{\RegionMarkerTok}[1]{{#1}}
    \newcommand{\ErrorTok}[1]{\textcolor[rgb]{1.00,0.00,0.00}{\textbf{{#1}}}}
    \newcommand{\NormalTok}[1]{{#1}}
    
    % Additional commands for more recent versions of Pandoc
    \newcommand{\ConstantTok}[1]{\textcolor[rgb]{0.53,0.00,0.00}{{#1}}}
    \newcommand{\SpecialCharTok}[1]{\textcolor[rgb]{0.25,0.44,0.63}{{#1}}}
    \newcommand{\VerbatimStringTok}[1]{\textcolor[rgb]{0.25,0.44,0.63}{{#1}}}
    \newcommand{\SpecialStringTok}[1]{\textcolor[rgb]{0.73,0.40,0.53}{{#1}}}
    \newcommand{\ImportTok}[1]{{#1}}
    \newcommand{\DocumentationTok}[1]{\textcolor[rgb]{0.73,0.13,0.13}{\textit{{#1}}}}
    \newcommand{\AnnotationTok}[1]{\textcolor[rgb]{0.38,0.63,0.69}{\textbf{\textit{{#1}}}}}
    \newcommand{\CommentVarTok}[1]{\textcolor[rgb]{0.38,0.63,0.69}{\textbf{\textit{{#1}}}}}
    \newcommand{\VariableTok}[1]{\textcolor[rgb]{0.10,0.09,0.49}{{#1}}}
    \newcommand{\ControlFlowTok}[1]{\textcolor[rgb]{0.00,0.44,0.13}{\textbf{{#1}}}}
    \newcommand{\OperatorTok}[1]{\textcolor[rgb]{0.40,0.40,0.40}{{#1}}}
    \newcommand{\BuiltInTok}[1]{{#1}}
    \newcommand{\ExtensionTok}[1]{{#1}}
    \newcommand{\PreprocessorTok}[1]{\textcolor[rgb]{0.74,0.48,0.00}{{#1}}}
    \newcommand{\AttributeTok}[1]{\textcolor[rgb]{0.49,0.56,0.16}{{#1}}}
    \newcommand{\InformationTok}[1]{\textcolor[rgb]{0.38,0.63,0.69}{\textbf{\textit{{#1}}}}}
    \newcommand{\WarningTok}[1]{\textcolor[rgb]{0.38,0.63,0.69}{\textbf{\textit{{#1}}}}}
    
    
    % Define a nice break command that doesn't care if a line doesn't already
    % exist.
    \def\br{\hspace*{\fill} \\* }
    % Math Jax compatability definitions
    \def\gt{>}
    \def\lt{<}
    % Document parameters
    \title{PrimeiroExercicio}
    
    
    

    % Pygments definitions
    
\makeatletter
\def\PY@reset{\let\PY@it=\relax \let\PY@bf=\relax%
    \let\PY@ul=\relax \let\PY@tc=\relax%
    \let\PY@bc=\relax \let\PY@ff=\relax}
\def\PY@tok#1{\csname PY@tok@#1\endcsname}
\def\PY@toks#1+{\ifx\relax#1\empty\else%
    \PY@tok{#1}\expandafter\PY@toks\fi}
\def\PY@do#1{\PY@bc{\PY@tc{\PY@ul{%
    \PY@it{\PY@bf{\PY@ff{#1}}}}}}}
\def\PY#1#2{\PY@reset\PY@toks#1+\relax+\PY@do{#2}}

\expandafter\def\csname PY@tok@w\endcsname{\def\PY@tc##1{\textcolor[rgb]{0.73,0.73,0.73}{##1}}}
\expandafter\def\csname PY@tok@c\endcsname{\let\PY@it=\textit\def\PY@tc##1{\textcolor[rgb]{0.25,0.50,0.50}{##1}}}
\expandafter\def\csname PY@tok@cp\endcsname{\def\PY@tc##1{\textcolor[rgb]{0.74,0.48,0.00}{##1}}}
\expandafter\def\csname PY@tok@k\endcsname{\let\PY@bf=\textbf\def\PY@tc##1{\textcolor[rgb]{0.00,0.50,0.00}{##1}}}
\expandafter\def\csname PY@tok@kp\endcsname{\def\PY@tc##1{\textcolor[rgb]{0.00,0.50,0.00}{##1}}}
\expandafter\def\csname PY@tok@kt\endcsname{\def\PY@tc##1{\textcolor[rgb]{0.69,0.00,0.25}{##1}}}
\expandafter\def\csname PY@tok@o\endcsname{\def\PY@tc##1{\textcolor[rgb]{0.40,0.40,0.40}{##1}}}
\expandafter\def\csname PY@tok@ow\endcsname{\let\PY@bf=\textbf\def\PY@tc##1{\textcolor[rgb]{0.67,0.13,1.00}{##1}}}
\expandafter\def\csname PY@tok@nb\endcsname{\def\PY@tc##1{\textcolor[rgb]{0.00,0.50,0.00}{##1}}}
\expandafter\def\csname PY@tok@nf\endcsname{\def\PY@tc##1{\textcolor[rgb]{0.00,0.00,1.00}{##1}}}
\expandafter\def\csname PY@tok@nc\endcsname{\let\PY@bf=\textbf\def\PY@tc##1{\textcolor[rgb]{0.00,0.00,1.00}{##1}}}
\expandafter\def\csname PY@tok@nn\endcsname{\let\PY@bf=\textbf\def\PY@tc##1{\textcolor[rgb]{0.00,0.00,1.00}{##1}}}
\expandafter\def\csname PY@tok@ne\endcsname{\let\PY@bf=\textbf\def\PY@tc##1{\textcolor[rgb]{0.82,0.25,0.23}{##1}}}
\expandafter\def\csname PY@tok@nv\endcsname{\def\PY@tc##1{\textcolor[rgb]{0.10,0.09,0.49}{##1}}}
\expandafter\def\csname PY@tok@no\endcsname{\def\PY@tc##1{\textcolor[rgb]{0.53,0.00,0.00}{##1}}}
\expandafter\def\csname PY@tok@nl\endcsname{\def\PY@tc##1{\textcolor[rgb]{0.63,0.63,0.00}{##1}}}
\expandafter\def\csname PY@tok@ni\endcsname{\let\PY@bf=\textbf\def\PY@tc##1{\textcolor[rgb]{0.60,0.60,0.60}{##1}}}
\expandafter\def\csname PY@tok@na\endcsname{\def\PY@tc##1{\textcolor[rgb]{0.49,0.56,0.16}{##1}}}
\expandafter\def\csname PY@tok@nt\endcsname{\let\PY@bf=\textbf\def\PY@tc##1{\textcolor[rgb]{0.00,0.50,0.00}{##1}}}
\expandafter\def\csname PY@tok@nd\endcsname{\def\PY@tc##1{\textcolor[rgb]{0.67,0.13,1.00}{##1}}}
\expandafter\def\csname PY@tok@s\endcsname{\def\PY@tc##1{\textcolor[rgb]{0.73,0.13,0.13}{##1}}}
\expandafter\def\csname PY@tok@sd\endcsname{\let\PY@it=\textit\def\PY@tc##1{\textcolor[rgb]{0.73,0.13,0.13}{##1}}}
\expandafter\def\csname PY@tok@si\endcsname{\let\PY@bf=\textbf\def\PY@tc##1{\textcolor[rgb]{0.73,0.40,0.53}{##1}}}
\expandafter\def\csname PY@tok@se\endcsname{\let\PY@bf=\textbf\def\PY@tc##1{\textcolor[rgb]{0.73,0.40,0.13}{##1}}}
\expandafter\def\csname PY@tok@sr\endcsname{\def\PY@tc##1{\textcolor[rgb]{0.73,0.40,0.53}{##1}}}
\expandafter\def\csname PY@tok@ss\endcsname{\def\PY@tc##1{\textcolor[rgb]{0.10,0.09,0.49}{##1}}}
\expandafter\def\csname PY@tok@sx\endcsname{\def\PY@tc##1{\textcolor[rgb]{0.00,0.50,0.00}{##1}}}
\expandafter\def\csname PY@tok@m\endcsname{\def\PY@tc##1{\textcolor[rgb]{0.40,0.40,0.40}{##1}}}
\expandafter\def\csname PY@tok@gh\endcsname{\let\PY@bf=\textbf\def\PY@tc##1{\textcolor[rgb]{0.00,0.00,0.50}{##1}}}
\expandafter\def\csname PY@tok@gu\endcsname{\let\PY@bf=\textbf\def\PY@tc##1{\textcolor[rgb]{0.50,0.00,0.50}{##1}}}
\expandafter\def\csname PY@tok@gd\endcsname{\def\PY@tc##1{\textcolor[rgb]{0.63,0.00,0.00}{##1}}}
\expandafter\def\csname PY@tok@gi\endcsname{\def\PY@tc##1{\textcolor[rgb]{0.00,0.63,0.00}{##1}}}
\expandafter\def\csname PY@tok@gr\endcsname{\def\PY@tc##1{\textcolor[rgb]{1.00,0.00,0.00}{##1}}}
\expandafter\def\csname PY@tok@ge\endcsname{\let\PY@it=\textit}
\expandafter\def\csname PY@tok@gs\endcsname{\let\PY@bf=\textbf}
\expandafter\def\csname PY@tok@gp\endcsname{\let\PY@bf=\textbf\def\PY@tc##1{\textcolor[rgb]{0.00,0.00,0.50}{##1}}}
\expandafter\def\csname PY@tok@go\endcsname{\def\PY@tc##1{\textcolor[rgb]{0.53,0.53,0.53}{##1}}}
\expandafter\def\csname PY@tok@gt\endcsname{\def\PY@tc##1{\textcolor[rgb]{0.00,0.27,0.87}{##1}}}
\expandafter\def\csname PY@tok@err\endcsname{\def\PY@bc##1{\setlength{\fboxsep}{0pt}\fcolorbox[rgb]{1.00,0.00,0.00}{1,1,1}{\strut ##1}}}
\expandafter\def\csname PY@tok@kc\endcsname{\let\PY@bf=\textbf\def\PY@tc##1{\textcolor[rgb]{0.00,0.50,0.00}{##1}}}
\expandafter\def\csname PY@tok@kd\endcsname{\let\PY@bf=\textbf\def\PY@tc##1{\textcolor[rgb]{0.00,0.50,0.00}{##1}}}
\expandafter\def\csname PY@tok@kn\endcsname{\let\PY@bf=\textbf\def\PY@tc##1{\textcolor[rgb]{0.00,0.50,0.00}{##1}}}
\expandafter\def\csname PY@tok@kr\endcsname{\let\PY@bf=\textbf\def\PY@tc##1{\textcolor[rgb]{0.00,0.50,0.00}{##1}}}
\expandafter\def\csname PY@tok@bp\endcsname{\def\PY@tc##1{\textcolor[rgb]{0.00,0.50,0.00}{##1}}}
\expandafter\def\csname PY@tok@fm\endcsname{\def\PY@tc##1{\textcolor[rgb]{0.00,0.00,1.00}{##1}}}
\expandafter\def\csname PY@tok@vc\endcsname{\def\PY@tc##1{\textcolor[rgb]{0.10,0.09,0.49}{##1}}}
\expandafter\def\csname PY@tok@vg\endcsname{\def\PY@tc##1{\textcolor[rgb]{0.10,0.09,0.49}{##1}}}
\expandafter\def\csname PY@tok@vi\endcsname{\def\PY@tc##1{\textcolor[rgb]{0.10,0.09,0.49}{##1}}}
\expandafter\def\csname PY@tok@vm\endcsname{\def\PY@tc##1{\textcolor[rgb]{0.10,0.09,0.49}{##1}}}
\expandafter\def\csname PY@tok@sa\endcsname{\def\PY@tc##1{\textcolor[rgb]{0.73,0.13,0.13}{##1}}}
\expandafter\def\csname PY@tok@sb\endcsname{\def\PY@tc##1{\textcolor[rgb]{0.73,0.13,0.13}{##1}}}
\expandafter\def\csname PY@tok@sc\endcsname{\def\PY@tc##1{\textcolor[rgb]{0.73,0.13,0.13}{##1}}}
\expandafter\def\csname PY@tok@dl\endcsname{\def\PY@tc##1{\textcolor[rgb]{0.73,0.13,0.13}{##1}}}
\expandafter\def\csname PY@tok@s2\endcsname{\def\PY@tc##1{\textcolor[rgb]{0.73,0.13,0.13}{##1}}}
\expandafter\def\csname PY@tok@sh\endcsname{\def\PY@tc##1{\textcolor[rgb]{0.73,0.13,0.13}{##1}}}
\expandafter\def\csname PY@tok@s1\endcsname{\def\PY@tc##1{\textcolor[rgb]{0.73,0.13,0.13}{##1}}}
\expandafter\def\csname PY@tok@mb\endcsname{\def\PY@tc##1{\textcolor[rgb]{0.40,0.40,0.40}{##1}}}
\expandafter\def\csname PY@tok@mf\endcsname{\def\PY@tc##1{\textcolor[rgb]{0.40,0.40,0.40}{##1}}}
\expandafter\def\csname PY@tok@mh\endcsname{\def\PY@tc##1{\textcolor[rgb]{0.40,0.40,0.40}{##1}}}
\expandafter\def\csname PY@tok@mi\endcsname{\def\PY@tc##1{\textcolor[rgb]{0.40,0.40,0.40}{##1}}}
\expandafter\def\csname PY@tok@il\endcsname{\def\PY@tc##1{\textcolor[rgb]{0.40,0.40,0.40}{##1}}}
\expandafter\def\csname PY@tok@mo\endcsname{\def\PY@tc##1{\textcolor[rgb]{0.40,0.40,0.40}{##1}}}
\expandafter\def\csname PY@tok@ch\endcsname{\let\PY@it=\textit\def\PY@tc##1{\textcolor[rgb]{0.25,0.50,0.50}{##1}}}
\expandafter\def\csname PY@tok@cm\endcsname{\let\PY@it=\textit\def\PY@tc##1{\textcolor[rgb]{0.25,0.50,0.50}{##1}}}
\expandafter\def\csname PY@tok@cpf\endcsname{\let\PY@it=\textit\def\PY@tc##1{\textcolor[rgb]{0.25,0.50,0.50}{##1}}}
\expandafter\def\csname PY@tok@c1\endcsname{\let\PY@it=\textit\def\PY@tc##1{\textcolor[rgb]{0.25,0.50,0.50}{##1}}}
\expandafter\def\csname PY@tok@cs\endcsname{\let\PY@it=\textit\def\PY@tc##1{\textcolor[rgb]{0.25,0.50,0.50}{##1}}}

\def\PYZbs{\char`\\}
\def\PYZus{\char`\_}
\def\PYZob{\char`\{}
\def\PYZcb{\char`\}}
\def\PYZca{\char`\^}
\def\PYZam{\char`\&}
\def\PYZlt{\char`\<}
\def\PYZgt{\char`\>}
\def\PYZsh{\char`\#}
\def\PYZpc{\char`\%}
\def\PYZdl{\char`\$}
\def\PYZhy{\char`\-}
\def\PYZsq{\char`\'}
\def\PYZdq{\char`\"}
\def\PYZti{\char`\~}
% for compatibility with earlier versions
\def\PYZat{@}
\def\PYZlb{[}
\def\PYZrb{]}
\makeatother


    % Exact colors from NB
    \definecolor{incolor}{rgb}{0.0, 0.0, 0.5}
    \definecolor{outcolor}{rgb}{0.545, 0.0, 0.0}



    
    % Prevent overflowing lines due to hard-to-break entities
    \sloppy 
    % Setup hyperref package
    \hypersetup{
      breaklinks=true,  % so long urls are correctly broken across lines
      colorlinks=true,
      urlcolor=urlcolor,
      linkcolor=linkcolor,
      citecolor=citecolor,
      }
    % Slightly bigger margins than the latex defaults
    
    \geometry{verbose,tmargin=1in,bmargin=1in,lmargin=1in,rmargin=1in}
    
    

    \begin{document}
    
    
    \maketitle
    
    

    
    \section{Primeiro Exercício}\label{primeiro-exercuxedcio}

\subsubsection{Aluno: Matheus Jun Ota}\label{aluno-matheus-jun-ota}

\subsubsection{RA: 138889}\label{ra-138889}

\begin{quote}
Fazer um programa de simulação e coletar dados para plotar um gráfico
retardo médio x utilização. A utilização é a razão entre a taxa de
chegada e a taxa de serviço do servidor. Assuma que o intervalo entre a
chegada de pacotes é exponencialmente distribuído, bem como o tamanho
dos pacotes. Entregar relatório com gráfico, código fonte e explicação
de como derivou intervalo de confiança e qual foi o critério adotado
para eliminar o transiente da simulação.
\end{quote}

Queremos fazer a simulação de uma fila M/M/1 onde o a taxa de chegada e
a taxa de serviço (proporcional ao tamanho do pacote) seguem a
distribuição exponencial. Para isso utilizaremos a linguagem Python, e o
relatório será escrito no Jupyter Notebook.

    \begin{Verbatim}[commandchars=\\\{\}]
{\color{incolor}In [{\color{incolor}1}]:} \PY{k+kn}{import} \PY{n+nn}{numpy} \PY{k}{as} \PY{n+nn}{np}
        \PY{k+kn}{import} \PY{n+nn}{pandas} \PY{k}{as} \PY{n+nn}{pd}
        \PY{k+kn}{import} \PY{n+nn}{seaborn} \PY{k}{as} \PY{n+nn}{sns}
        \PY{k+kn}{import} \PY{n+nn}{itertools}
        \PY{k+kn}{import} \PY{n+nn}{matplotlib}\PY{n+nn}{.}\PY{n+nn}{pyplot} \PY{k}{as} \PY{n+nn}{plt}
        \PY{k+kn}{import} \PY{n+nn}{math}
        \PY{k+kn}{import} \PY{n+nn}{random}
        \PY{k+kn}{import} \PY{n+nn}{statistics} 
        \PY{k+kn}{from} \PY{n+nn}{collections} \PY{k}{import} \PY{n}{deque}
        
        \PY{k}{class} \PY{n+nc}{Queue}\PY{p}{(}\PY{p}{)}\PY{p}{:}
            \PY{k}{def} \PY{n+nf}{\PYZus{}\PYZus{}init\PYZus{}\PYZus{}}\PY{p}{(}\PY{n+nb+bp}{self}\PY{p}{,} \PY{n}{arrivalRate}\PY{p}{,} \PY{n}{departureRate}\PY{p}{)}\PY{p}{:}
                \PY{n+nb+bp}{self}\PY{o}{.}\PY{n}{usersInQueue} \PY{o}{=} \PY{l+m+mi}{0}
                \PY{n+nb+bp}{self}\PY{o}{.}\PY{n}{servicedUsers} \PY{o}{=} \PY{l+m+mi}{0}
                \PY{n+nb+bp}{self}\PY{o}{.}\PY{n}{currTime} \PY{o}{=} \PY{l+m+mi}{0}
                \PY{n+nb+bp}{self}\PY{o}{.}\PY{n}{arrivalParam} \PY{o}{=} \PY{n}{arrivalRate}
                \PY{n+nb+bp}{self}\PY{o}{.}\PY{n}{departureParam} \PY{o}{=} \PY{n}{departureRate}
                \PY{n+nb+bp}{self}\PY{o}{.}\PY{n}{arrivalTime} \PY{o}{=} \PY{n+nb+bp}{self}\PY{o}{.}\PY{n}{getRandomTime}\PY{p}{(}\PY{n+nb+bp}{self}\PY{o}{.}\PY{n}{arrivalParam}\PY{p}{)}
                \PY{n+nb+bp}{self}\PY{o}{.}\PY{n}{inQueueArrivals} \PY{o}{=} \PY{n}{deque}\PY{p}{(}\PY{p}{)}
                \PY{n+nb+bp}{self}\PY{o}{.}\PY{n}{departureTime} \PY{o}{=} \PY{n+nb}{float}\PY{p}{(}\PY{l+s+s2}{\PYZdq{}}\PY{l+s+s2}{inf}\PY{l+s+s2}{\PYZdq{}}\PY{p}{)}
                \PY{n+nb+bp}{self}\PY{o}{.}\PY{n}{totalDelay} \PY{o}{=} \PY{l+m+mi}{0}
\end{Verbatim}


    A fim de facilitar a simulação, iremos proceder com uma abordagem
baseada em orientação a objetos. Assim, começamos por declarar uma
classe \textbf{Queue} que possui um construtor que recebe como
parâmetros a taxa de chegada e a taxa de serviço. Os atributos dessa
classe são os seguintes:

\textbf{usersInQueue}: número de usuários na fila

\textbf{currTime}: tempo corrente na simulação da fila

\textbf{arrivalParam}: parâmetro utilizado na distribuição exponencial
para obter o intervalo entre chegada de pacotes

\textbf{departureParam}: parâmetro utilizado na distribuição exponencial
para obter o tempo de serviço do pacote

\textbf{arrivalTime}: tempo para a chegada do próximo pacote

\textbf{inQueueArrivals}: lista com os tempos de chegada dos pacotes que
estão na fila

\textbf{departureTime}: tempo para a saída do próximo pacote

\textbf{totalDelay}: tempo total que os pacotes esperaram para serem
atendidos

    \begin{Verbatim}[commandchars=\\\{\}]
{\color{incolor}In [{\color{incolor}2}]:} \PY{k}{class} \PY{n+nc}{Queue}\PY{p}{(}\PY{n}{Queue}\PY{p}{)}\PY{p}{:}
            \PY{k}{def} \PY{n+nf}{getRandomTime}\PY{p}{(}\PY{n+nb+bp}{self}\PY{p}{,} \PY{n}{param}\PY{p}{)}\PY{p}{:}
                \PY{k}{return} \PY{n}{random}\PY{o}{.}\PY{n}{expovariate}\PY{p}{(}\PY{n}{param}\PY{p}{)}
\end{Verbatim}


    Essa função simplesmente retorna um número de acordo com a distribuição
exponencial com parâmetro dado por \textbf{param}. Como é de se esperar,
utilizaremos esse método para gerar os intervalos entre chegadas e o
tempo de serviço de cada pacote.

Obs.: Como não é possível escrever diferentes blocos de código para a
mesma classe no Jupyter Notebook, fizemos a "redeclaração" da classe
\textbf{Queue}, e herdamos dela mesma. Assim, temos uma classe
\textbf{Queue} com todos os métodos declarados para ela até agora.

    \begin{Verbatim}[commandchars=\\\{\}]
{\color{incolor}In [{\color{incolor}3}]:} \PY{k}{class} \PY{n+nc}{Queue}\PY{p}{(}\PY{n}{Queue}\PY{p}{)}\PY{p}{:}
            \PY{k}{def} \PY{n+nf}{goToNextEvent}\PY{p}{(}\PY{n+nb+bp}{self}\PY{p}{)}\PY{p}{:}
                \PY{k}{if} \PY{n+nb+bp}{self}\PY{o}{.}\PY{n}{arrivalTime} \PY{o}{\PYZlt{}} \PY{n+nb+bp}{self}\PY{o}{.}\PY{n}{departureTime}\PY{p}{:}
                    \PY{n+nb+bp}{self}\PY{o}{.}\PY{n}{currTime} \PY{o}{=} \PY{n+nb+bp}{self}\PY{o}{.}\PY{n}{arrivalTime}
                    \PY{n+nb+bp}{self}\PY{o}{.}\PY{n}{handleArrivalEvent}\PY{p}{(}\PY{p}{)}
                \PY{k}{else}\PY{p}{:}
                    \PY{n+nb+bp}{self}\PY{o}{.}\PY{n}{currTime} \PY{o}{=} \PY{n+nb+bp}{self}\PY{o}{.}\PY{n}{departureTime}
                    \PY{n+nb+bp}{self}\PY{o}{.}\PY{n}{handleDepartureEvent}\PY{p}{(}\PY{p}{)}
\end{Verbatim}


    Aqui definimos um método para avançar o tempo corrente da fila para o
próximo evento (chegada ou saída de cliente) e tratá-lo de acordo.

    \begin{Verbatim}[commandchars=\\\{\}]
{\color{incolor}In [{\color{incolor}4}]:} \PY{k}{class} \PY{n+nc}{Queue}\PY{p}{(}\PY{n}{Queue}\PY{p}{)}\PY{p}{:}
            \PY{k}{def} \PY{n+nf}{handleArrivalEvent}\PY{p}{(}\PY{n+nb+bp}{self}\PY{p}{)}\PY{p}{:}
                \PY{n+nb+bp}{self}\PY{o}{.}\PY{n}{usersInQueue} \PY{o}{+}\PY{o}{=} \PY{l+m+mi}{1}
        
                \PY{k}{if} \PY{n+nb+bp}{self}\PY{o}{.}\PY{n}{usersInQueue} \PY{o}{\PYZlt{}}\PY{o}{=} \PY{l+m+mi}{1}\PY{p}{:}
                    \PY{n+nb+bp}{self}\PY{o}{.}\PY{n}{departureTime} \PY{o}{=} \PY{n+nb+bp}{self}\PY{o}{.}\PY{n}{currTime} \PY{o}{+} \PY{n+nb+bp}{self}\PY{o}{.}\PY{n}{getRandomTime}\PY{p}{(}\PY{n+nb+bp}{self}\PY{o}{.}\PY{n}{departureParam}\PY{p}{)}
        
                \PY{n+nb+bp}{self}\PY{o}{.}\PY{n}{inQueueArrivals}\PY{o}{.}\PY{n}{append}\PY{p}{(}\PY{n+nb+bp}{self}\PY{o}{.}\PY{n}{arrivalTime}\PY{p}{)}
                \PY{n+nb+bp}{self}\PY{o}{.}\PY{n}{arrivalTime} \PY{o}{=} \PY{n+nb+bp}{self}\PY{o}{.}\PY{n}{currTime} \PY{o}{+} \PY{n+nb+bp}{self}\PY{o}{.}\PY{n}{getRandomTime}\PY{p}{(}\PY{n+nb+bp}{self}\PY{o}{.}\PY{n}{arrivalParam}\PY{p}{)}
\end{Verbatim}


    Nesse método fazemos o tratamento de um evento de chegada de cliente.
Seja \(t_s\) uma função que mapeia cada cliente para o seu tempo de
saída e \(t_c\) uma função que mapeia para o tempo de chegada, se após o
evento de chegada de cliente, a fila possuir somente um usuário \(u_1\)
na fila, isso significa que o próximo tempo de saída deve ser
correspondente ao tempo de saída de \(u_1\), isto é, \(t_s(u_1)\). Pois
para qualquer outro usuário \(u_2\) que chegue após \(u_1\), teremos que
\(t_s(u_2) \geq t_s(u_1)\), pois \(t_c(u_2) \geq t_c(u_1)\) e a fila é
do tipo FIFO (First-In-First-Out).

Obs.: Note que ao invés de gerar uma lista de eventos de chegada e de
saída, e simular esses eventos na fila; nós nos preocupamos somente com
o tempo do próximo evento de chegada/saída de pacote. Isso pode ser
feito pois, seja \(C = [t_c(u_1), ..., t_c(u_n)]\) uma lista ordenada em
ordem crescente, então \(S = [t_s(u_1), ..., t_s(u_n)]\) também está
ordenada em ordem crescente, pois a fila é do tipo FIFO. Logo, não
precisamos ter uma lista de eventos com os tempos de chegada/saída,
basta que mantenhamos a invariante de que \textbf{departureTime} seja
sempre um tempo de saída de um cliente que está na fila e que chegou
antes dos outros. Por outro lado, precisamos da lista
\textbf{inQueueArrivals} para calcular o delay experienciado por cada
pacote.

    \begin{Verbatim}[commandchars=\\\{\}]
{\color{incolor}In [{\color{incolor}5}]:} \PY{k}{class} \PY{n+nc}{Queue}\PY{p}{(}\PY{n}{Queue}\PY{p}{)}\PY{p}{:}
            \PY{k}{def} \PY{n+nf}{handleDepartureEvent}\PY{p}{(}\PY{n+nb+bp}{self}\PY{p}{)}\PY{p}{:}
                \PY{n+nb+bp}{self}\PY{o}{.}\PY{n}{usersInQueue} \PY{o}{\PYZhy{}}\PY{o}{=} \PY{l+m+mi}{1}
                \PY{n+nb+bp}{self}\PY{o}{.}\PY{n}{servicedUsers} \PY{o}{+}\PY{o}{=} \PY{l+m+mi}{1}
                
                \PY{c+c1}{\PYZsh{} add the delay of this packet and remove it from the inQueueArrivals list}
                \PY{n+nb+bp}{self}\PY{o}{.}\PY{n}{totalDelay} \PY{o}{+}\PY{o}{=} \PY{p}{(}\PY{n+nb+bp}{self}\PY{o}{.}\PY{n}{departureTime} \PY{o}{\PYZhy{}} \PY{n+nb+bp}{self}\PY{o}{.}\PY{n}{inQueueArrivals}\PY{p}{[}\PY{l+m+mi}{0}\PY{p}{]}\PY{p}{)}
                \PY{n+nb+bp}{self}\PY{o}{.}\PY{n}{inQueueArrivals}\PY{o}{.}\PY{n}{popleft}\PY{p}{(}\PY{p}{)}
                
                \PY{k}{if} \PY{n+nb+bp}{self}\PY{o}{.}\PY{n}{usersInQueue} \PY{o}{\PYZgt{}} \PY{l+m+mi}{0}\PY{p}{:}
                    \PY{n+nb+bp}{self}\PY{o}{.}\PY{n}{departureTime} \PY{o}{=} \PY{n+nb+bp}{self}\PY{o}{.}\PY{n}{currTime} \PY{o}{+} \PY{n+nb+bp}{self}\PY{o}{.}\PY{n}{getRandomTime}\PY{p}{(}\PY{n+nb+bp}{self}\PY{o}{.}\PY{n}{departureParam}\PY{p}{)}
                \PY{k}{else}\PY{p}{:}
                    \PY{n+nb+bp}{self}\PY{o}{.}\PY{n}{departureTime} \PY{o}{=} \PY{n+nb}{float}\PY{p}{(}\PY{l+s+s2}{\PYZdq{}}\PY{l+s+s2}{inf}\PY{l+s+s2}{\PYZdq{}}\PY{p}{)}
\end{Verbatim}


    Semelhantemente ao método acima, nesse método fazemos o tratamento de um
evento de saída de cliente. Se após a saída do cliente, a fila está
vazia, então não existe um próximo tempo de saída (\textbf{departureTime
= \(\infty\)}), caso contrário, temos que setar um novo tempo de saída
para o pacote que está "na frente" da fila.

    \begin{Verbatim}[commandchars=\\\{\}]
{\color{incolor}In [{\color{incolor}6}]:} \PY{k}{class} \PY{n+nc}{Queue}\PY{p}{(}\PY{n}{Queue}\PY{p}{)}\PY{p}{:}
            \PY{k}{def} \PY{n+nf}{getMeanDelay}\PY{p}{(}\PY{n+nb+bp}{self}\PY{p}{)}\PY{p}{:}
                \PY{k}{if} \PY{n+nb+bp}{self}\PY{o}{.}\PY{n}{servicedUsers} \PY{o}{==} \PY{l+m+mi}{0}\PY{p}{:}
                    \PY{k}{return} \PY{l+m+mi}{0}
                \PY{k}{else}\PY{p}{:}
                    \PY{k}{return} \PY{n+nb+bp}{self}\PY{o}{.}\PY{n}{totalDelay} \PY{o}{/} \PY{n+nb+bp}{self}\PY{o}{.}\PY{n}{servicedUsers}
\end{Verbatim}


    Por fim, esse método simplesmente calcula o retardo médio experienciado
pelos pacotes.

Agora que temos a classe \textbf{Queue}, afim de obter de maneira fácil
o intervalo de confiança das medidas que serão feitas, vamos criar
também uma classe \textbf{QueueStatistics} que irá fornecer um método
para obter a média aritmética de uma lista; e outro para calcular o erro
de uma amostra, que é a variação utilizada no intervalo de confiança,
dada por \(\frac{1.96}{\sqrt{n}} * \sigma\). Onde \(\sigma\) denota o
desvio padrão da amostra.

    \begin{Verbatim}[commandchars=\\\{\}]
{\color{incolor}In [{\color{incolor}7}]:} \PY{k}{class} \PY{n+nc}{QueueStatistics}\PY{p}{(}\PY{p}{)}\PY{p}{:}
            \PY{k}{def} \PY{n+nf}{getMean}\PY{p}{(}\PY{n+nb+bp}{self}\PY{p}{,} \PY{n}{L}\PY{p}{)}\PY{p}{:}
                \PY{k}{return} \PY{n+nb}{sum}\PY{p}{(}\PY{n}{L}\PY{p}{)} \PY{o}{/} \PY{n+nb}{len}\PY{p}{(}\PY{n}{L}\PY{p}{)}
        
            \PY{k}{def} \PY{n+nf}{getErr}\PY{p}{(}\PY{n+nb+bp}{self}\PY{p}{,} \PY{n}{L}\PY{p}{)}\PY{p}{:}
                \PY{k}{return} \PY{p}{(}\PY{l+m+mf}{1.96} \PY{o}{/} \PY{n}{math}\PY{o}{.}\PY{n}{sqrt}\PY{p}{(}\PY{n+nb}{len}\PY{p}{(}\PY{n}{L}\PY{p}{)}\PY{p}{)}\PY{p}{)} \PY{o}{*} \PY{n}{statistics}\PY{o}{.}\PY{n}{stdev}\PY{p}{(}\PY{n}{L}\PY{p}{)}
\end{Verbatim}


    Agora que temos as classes \textbf{Queue} e \textbf{QueueStatistics},
queremos descobrir como remover o \textbf{transiente} da simulação. Para
isso, iremos executar a simulação algumas vezes e plotar um gráfico de
retardo médio pelo número de eventos simulados.

    \begin{Verbatim}[commandchars=\\\{\}]
{\color{incolor}In [{\color{incolor}8}]:} \PY{c+c1}{\PYZsh{} run a bunch of simulations and collect mean delays and the errors}
        \PY{n}{qs} \PY{o}{=} \PY{n}{QueueStatistics}\PY{p}{(}\PY{p}{)}
        \PY{n}{means} \PY{o}{=} \PY{p}{[}\PY{p}{]}
        \PY{n}{errs} \PY{o}{=} \PY{p}{[}\PY{p}{]}
        \PY{n}{events} \PY{o}{=} \PY{p}{[}\PY{p}{]}
        \PY{n}{queues} \PY{o}{=} \PY{p}{[}\PY{n}{Queue}\PY{p}{(}\PY{l+m+mf}{0.25}\PY{p}{,} \PY{l+m+mf}{0.5}\PY{p}{)} \PY{k}{for} \PY{n}{\PYZus{}} \PY{o+ow}{in} \PY{n+nb}{range}\PY{p}{(}\PY{l+m+mi}{100}\PY{p}{)}\PY{p}{]}
        \PY{k}{for} \PY{n}{i} \PY{o+ow}{in} \PY{n+nb}{range}\PY{p}{(}\PY{l+m+mi}{1001}\PY{p}{)}\PY{p}{:} \PY{c+c1}{\PYZsh{} we will run 1000 events in each queue}
            \PY{n}{L} \PY{o}{=} \PY{p}{[}\PY{p}{]}
        
            \PY{c+c1}{\PYZsh{} advance each queue}
            \PY{k}{for} \PY{n}{j} \PY{o+ow}{in} \PY{n+nb}{range}\PY{p}{(}\PY{n+nb}{len}\PY{p}{(}\PY{n}{queues}\PY{p}{)}\PY{p}{)}\PY{p}{:}
                \PY{n}{queues}\PY{p}{[}\PY{n}{j}\PY{p}{]}\PY{o}{.}\PY{n}{goToNextEvent}\PY{p}{(}\PY{p}{)}
            
            \PY{c+c1}{\PYZsh{} collect metrics}
            \PY{k}{if} \PY{n}{i} \PY{o}{\PYZpc{}} \PY{l+m+mi}{200} \PY{o}{==} \PY{l+m+mi}{0}\PY{p}{:}
                \PY{n}{L} \PY{o}{=} \PY{p}{[}\PY{p}{]}
                
                \PY{k}{for} \PY{n}{j} \PY{o+ow}{in} \PY{n+nb}{range}\PY{p}{(}\PY{n+nb}{len}\PY{p}{(}\PY{n}{queues}\PY{p}{)}\PY{p}{)}\PY{p}{:}
                    \PY{n}{L}\PY{o}{.}\PY{n}{append}\PY{p}{(}\PY{n}{queues}\PY{p}{[}\PY{n}{j}\PY{p}{]}\PY{o}{.}\PY{n}{getMeanDelay}\PY{p}{(}\PY{p}{)}\PY{p}{)}
            
                \PY{n}{means}\PY{o}{.}\PY{n}{append}\PY{p}{(}\PY{n}{qs}\PY{o}{.}\PY{n}{getMean}\PY{p}{(}\PY{n}{L}\PY{p}{)}\PY{p}{)}
                \PY{n}{errs}\PY{o}{.}\PY{n}{append}\PY{p}{(}\PY{n}{qs}\PY{o}{.}\PY{n}{getErr}\PY{p}{(}\PY{n}{L}\PY{p}{)}\PY{p}{)}
                \PY{n}{events}\PY{o}{.}\PY{n}{append}\PY{p}{(}\PY{n}{i}\PY{p}{)}
        
        \PY{c+c1}{\PYZsh{} configure some plotting properties}
        \PY{n}{sns}\PY{o}{.}\PY{n}{set\PYZus{}style}\PY{p}{(}\PY{l+s+s2}{\PYZdq{}}\PY{l+s+s2}{darkgrid}\PY{l+s+s2}{\PYZdq{}}\PY{p}{)}
        \PY{n}{sns}\PY{o}{.}\PY{n}{set\PYZus{}palette}\PY{p}{(}\PY{l+s+s2}{\PYZdq{}}\PY{l+s+s2}{hls}\PY{l+s+s2}{\PYZdq{}}\PY{p}{)}
        \PY{n}{plt}\PY{o}{.}\PY{n}{rcParams}\PY{p}{[}\PY{l+s+s1}{\PYZsq{}}\PY{l+s+s1}{figure.figsize}\PY{l+s+s1}{\PYZsq{}}\PY{p}{]} \PY{o}{=} \PY{p}{[}\PY{l+m+mi}{15}\PY{p}{,} \PY{l+m+mi}{8}\PY{p}{]}
        
        \PY{c+c1}{\PYZsh{} create a pandas dataframe for the collected data}
        \PY{n}{aux} \PY{o}{=} \PY{p}{\PYZob{}}\PY{p}{\PYZcb{}}
        \PY{n}{aux}\PY{p}{[}\PY{l+s+s2}{\PYZdq{}}\PY{l+s+s2}{events executed}\PY{l+s+s2}{\PYZdq{}}\PY{p}{]} \PY{o}{=} \PY{n}{events}
        \PY{n}{aux}\PY{p}{[}\PY{l+s+s2}{\PYZdq{}}\PY{l+s+s2}{mean delay}\PY{l+s+s2}{\PYZdq{}}\PY{p}{]} \PY{o}{=} \PY{n}{means}
        \PY{n}{df} \PY{o}{=} \PY{n}{pd}\PY{o}{.}\PY{n}{DataFrame}\PY{p}{(}\PY{n}{aux}\PY{p}{)}
        
        \PY{c+c1}{\PYZsh{} plot the means using seaborn}
        \PY{n}{ax} \PY{o}{=} \PY{n}{sns}\PY{o}{.}\PY{n}{pointplot}\PY{p}{(}\PY{n}{x}\PY{o}{=}\PY{l+s+s2}{\PYZdq{}}\PY{l+s+s2}{events executed}\PY{l+s+s2}{\PYZdq{}}\PY{p}{,} \PY{n}{y}\PY{o}{=}\PY{l+s+s2}{\PYZdq{}}\PY{l+s+s2}{mean delay}\PY{l+s+s2}{\PYZdq{}}\PY{p}{,} \PY{n}{data}\PY{o}{=}\PY{n}{df}\PY{p}{,} \PY{n}{color}\PY{o}{=}\PY{l+s+s2}{\PYZdq{}}\PY{l+s+s2}{limegreen}\PY{l+s+s2}{\PYZdq{}}\PY{p}{,} \PY{n}{ci} \PY{o}{=} \PY{l+m+mi}{95}\PY{p}{)}
        
        \PY{c+c1}{\PYZsh{} plot confidence intervals using matplotlib}
        \PY{n}{ax}\PY{o}{.}\PY{n}{errorbar}\PY{p}{(}\PY{n}{df}\PY{o}{.}\PY{n}{index}\PY{p}{,} \PY{n}{means}\PY{p}{,} \PY{n}{yerr}\PY{o}{=}\PY{n}{errs}\PY{p}{,} \PY{n}{fmt}\PY{o}{=}\PY{l+s+s2}{\PYZdq{}}\PY{l+s+s2}{\PYZhy{}}\PY{l+s+s2}{\PYZdq{}}\PY{p}{,} \PY{n}{color}\PY{o}{=}\PY{l+s+s2}{\PYZdq{}}\PY{l+s+s2}{limegreen}\PY{l+s+s2}{\PYZdq{}}\PY{p}{,} \PY{n}{capsize} \PY{o}{=} \PY{l+m+mi}{5}\PY{p}{,} \PY{n}{capthick} \PY{o}{=} \PY{l+m+mi}{2}\PY{p}{,} \PY{n}{linewidth} \PY{o}{=} \PY{l+m+mi}{2}\PY{p}{)}
        
        \PY{n}{plt}\PY{o}{.}\PY{n}{show}\PY{p}{(}\PY{p}{)}
\end{Verbatim}


    \begin{center}
    \adjustimage{max size={0.9\linewidth}{0.9\paperheight}}{output_15_0.png}
    \end{center}
    { \hspace*{\fill} \\}
    
    Observando o gráfico, concluímos que a partir de 1000 eventos simulados,
o efeito do transiente é desprezível. Assim, coletaremos os dados da
simulação após ocorrer 1000 eventos.

Uma vez que descobrimos como remover o transiente da simulação, podemos
proceder para a criação do gráfico de retardo médio por utilização. Para
isso, iremos variar a taxa de chegada \textbf{arrivalRate} de 0.05 até
0.95 e utilizaremos o valor fixo de 1 para a taxa de saída
\textbf{departureRate}, obtendo assim, diferentes valores de utilização.
Para cada um desses valores de utilização, iremos executar 1000
simulações, e cada uma irá executar 1000 eventos.

    \begin{Verbatim}[commandchars=\\\{\}]
{\color{incolor}In [{\color{incolor}9}]:} \PY{n}{means} \PY{o}{=} \PY{p}{[}\PY{p}{]}
        \PY{n}{errs} \PY{o}{=} \PY{p}{[}\PY{p}{]}
        \PY{n}{utilization} \PY{o}{=} \PY{p}{[}\PY{p}{]}
        \PY{n}{departureRate} \PY{o}{=} \PY{l+m+mi}{1}
        
        \PY{c+c1}{\PYZsh{} we change the arrival rate to change the utilization}
        \PY{k}{for} \PY{n}{x} \PY{o+ow}{in} \PY{n+nb}{range}\PY{p}{(}\PY{l+m+mi}{5}\PY{p}{,} \PY{l+m+mi}{100}\PY{p}{,} \PY{l+m+mi}{5}\PY{p}{)}\PY{p}{:}
            \PY{n}{arrivalRate} \PY{o}{=} \PY{n}{x} \PY{o}{/} \PY{l+m+mf}{100.}
            \PY{n}{L} \PY{o}{=} \PY{p}{[}\PY{p}{]}
            
            \PY{c+c1}{\PYZsh{} run a bunch of simulations and collect mean delays and errors}
            \PY{k}{for} \PY{n}{j} \PY{o+ow}{in} \PY{n+nb}{range}\PY{p}{(}\PY{l+m+mi}{1000}\PY{p}{)}\PY{p}{:}
                \PY{n}{q} \PY{o}{=} \PY{n}{Queue}\PY{p}{(}\PY{n}{arrivalRate}\PY{p}{,} \PY{n}{departureRate}\PY{p}{)}
                
                \PY{k}{for} \PY{n}{\PYZus{}} \PY{o+ow}{in} \PY{n+nb}{range}\PY{p}{(}\PY{l+m+mi}{1000}\PY{p}{)}\PY{p}{:}
                    \PY{n}{q}\PY{o}{.}\PY{n}{goToNextEvent}\PY{p}{(}\PY{p}{)}
            
                \PY{n}{L}\PY{o}{.}\PY{n}{append}\PY{p}{(}\PY{n}{q}\PY{o}{.}\PY{n}{getMeanDelay}\PY{p}{(}\PY{p}{)}\PY{p}{)}
            
            \PY{n}{utilization}\PY{o}{.}\PY{n}{append}\PY{p}{(}\PY{n+nb}{round}\PY{p}{(}\PY{n}{arrivalRate} \PY{o}{/} \PY{n}{departureRate}\PY{p}{,} \PY{l+m+mi}{2}\PY{p}{)}\PY{p}{)}
            \PY{n}{means}\PY{o}{.}\PY{n}{append}\PY{p}{(}\PY{n}{qs}\PY{o}{.}\PY{n}{getMean}\PY{p}{(}\PY{n}{L}\PY{p}{)}\PY{p}{)}
            \PY{n}{errs}\PY{o}{.}\PY{n}{append}\PY{p}{(}\PY{n}{qs}\PY{o}{.}\PY{n}{getErr}\PY{p}{(}\PY{n}{L}\PY{p}{)}\PY{p}{)}
        
        \PY{c+c1}{\PYZsh{} create a pandas dataframe for the collected data}
        \PY{n}{aux} \PY{o}{=} \PY{p}{\PYZob{}}\PY{p}{\PYZcb{}}
        \PY{n}{aux}\PY{p}{[}\PY{l+s+s2}{\PYZdq{}}\PY{l+s+s2}{utilization}\PY{l+s+s2}{\PYZdq{}}\PY{p}{]} \PY{o}{=} \PY{n}{utilization}
        \PY{n}{aux}\PY{p}{[}\PY{l+s+s2}{\PYZdq{}}\PY{l+s+s2}{mean delay}\PY{l+s+s2}{\PYZdq{}}\PY{p}{]} \PY{o}{=} \PY{n}{means}
        \PY{n}{df} \PY{o}{=} \PY{n}{pd}\PY{o}{.}\PY{n}{DataFrame}\PY{p}{(}\PY{n}{aux}\PY{p}{)}
        
        \PY{c+c1}{\PYZsh{} plot the means using seaborn}
        \PY{n}{ax} \PY{o}{=} \PY{n}{sns}\PY{o}{.}\PY{n}{pointplot}\PY{p}{(}\PY{n}{x}\PY{o}{=}\PY{l+s+s2}{\PYZdq{}}\PY{l+s+s2}{utilization}\PY{l+s+s2}{\PYZdq{}}\PY{p}{,} \PY{n}{y}\PY{o}{=}\PY{l+s+s2}{\PYZdq{}}\PY{l+s+s2}{mean delay}\PY{l+s+s2}{\PYZdq{}}\PY{p}{,} \PY{n}{data}\PY{o}{=}\PY{n}{df}\PY{p}{,} \PY{n}{color}\PY{o}{=}\PY{l+s+s2}{\PYZdq{}}\PY{l+s+s2}{limegreen}\PY{l+s+s2}{\PYZdq{}}\PY{p}{,} \PY{n}{ci} \PY{o}{=} \PY{l+m+mi}{95}\PY{p}{)}
        
        \PY{c+c1}{\PYZsh{} plot confidence intervals using matplotlib}
        \PY{n}{ax}\PY{o}{.}\PY{n}{errorbar}\PY{p}{(}\PY{n}{df}\PY{o}{.}\PY{n}{index}\PY{p}{,} \PY{n}{means}\PY{p}{,} \PY{n}{yerr}\PY{o}{=}\PY{n}{errs}\PY{p}{,} \PY{n}{fmt}\PY{o}{=}\PY{l+s+s2}{\PYZdq{}}\PY{l+s+s2}{\PYZhy{}}\PY{l+s+s2}{\PYZdq{}}\PY{p}{,} \PY{n}{color}\PY{o}{=}\PY{l+s+s2}{\PYZdq{}}\PY{l+s+s2}{limegreen}\PY{l+s+s2}{\PYZdq{}}\PY{p}{,} \PY{n}{capsize} \PY{o}{=} \PY{l+m+mi}{3}\PY{p}{,} \PY{n}{elinewidth} \PY{o}{=} \PY{l+m+mi}{2}\PY{p}{)}
        
        \PY{n}{plt}\PY{o}{.}\PY{n}{show}\PY{p}{(}\PY{p}{)}
\end{Verbatim}


    \begin{center}
    \adjustimage{max size={0.9\linewidth}{0.9\paperheight}}{output_17_0.png}
    \end{center}
    { \hspace*{\fill} \\}
    
    E portanto, o gráfico obtido é condizente com o esperado.


    % Add a bibliography block to the postdoc
    
    
    
    \end{document}
